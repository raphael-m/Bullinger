\documentclass[12pt]{article}

\usepackage{geometry}
\geometry {
	a4paper,
	left=2cm,
	right=2cm,
	top=2.5cm,
	bottom=1.5cm
}

% Sprache, Kodierung
\usepackage[german]{babel}  % Silbentrennung ("Table of Contents" --> "Inhaltsverzeichnis")
\usepackage[utf8]{inputenc}  % Umlaute
\usepackage[T1]{fontenc}

% Farben
\usepackage{xcolor}
\usepackage[colorlinks=true, urlcolor=mailblue]{hyperref}

\definecolor{mailblue}{RGB}{6,69,173}
\definecolor{mygreen}{rgb}{0,0.6,0}
\definecolor{deepgreen}{rgb}{0,0.7,0}
\definecolor{mygray}{rgb}{0.5,0.5,0.5}
\definecolor{mymauve}{rgb}{0.58,0,0.82}
\definecolor{deepred}{rgb}{0.6,0,0}

\usepackage{array}
\usepackage{tikz}
\usetikzlibrary{positioning,automata}

% Abkürzungsverzeichnis
\usepackage{suffix}
\usepackage{xstring}
\usepackage[printonlyused, withpage]{acronym}
    
\usepackage{filecontents}
\usepackage{datatool}
\usepackage{amsmath}
\usepackage{amssymb}
\usepackage{mathabx}  % diameter (avg) symbol
\usepackage{textcomp}
\usepackage{tikz}
\usepackage{pgfgantt}
\usepackage{float}
\usepackage{calc}
\usepackage[symbol]{footmisc}
\usepackage{graphicx}

\usepackage{pgfplots}
\pgfplotsset{width=7cm, compat=1.5, table/search path={Data}}
% Preamble: \pgfplotsset{,compat=1.16}

\usepackage{caption}

% Main-Title
\newcommand\maintitle[2]{
	\begin{center}
		\textbf{{\Huge\textsc{#1}}}\\
		\vspace{-0.2cm}
		\rule{0.7\textwidth}{3pt}\\
		\vspace{0.2cm}
		{\large #2}\\
	\end{center}
}

% subsubsubsection
\newcommand
\ssss[1] {\medskip\noindent\textsc{\textbf{#1\medskip}}}

% Header
\newcommand\newtitle[1]{
	%\textrule{0.5pt}
	
	%\vspace{-0.3cm}
	{\noindent\bigskip\Large\bf\scshape #1}
	
	\vspace{-0.7cm}
	\textrule{0.5pt}
	\vspace{-0.5cm}
}

% Anführungs- und Schlusszeichen
\newcommand
\quotes[1] {\flqq #1\frqq}

% Horizontale Linie
\usepackage[normalem]{ulem}
\newcommand
\textrule[1] {
	\noindent
	\rule{\textwidth}{#1}
}

% Abbildung (Caption)
\newcommand
\abb[1] {
	\begin{quotation}
		\vspace{-0.8cm}
		\noindent\caption{#1}
	\end{quotation}
}

% Erster Buchstabe eines neuen Kapitels
\usepackage{lettrine}
\newcommand\fl[1] {
	\lettrine[lraise=0.05, nindent=0em, slope=-.5em]{#1}{}{}
}
\newcommand\fls[1] {
	\lettrine[lines=2]{#1}{}{}
}

% Listen
\renewcommand{\labelitemi}{$\blacktriangleright$}
\renewcommand{\labelitemii}{$\diamond$}
\renewcommand{\labelitemiii}{$\triangleright$}
\renewcommand{\labelitemiv}{$\ast$}
\newcommand{\litem}{\text{-}}  % Literaturverzeichnis
\usepackage{enumitem}

% TOC
\usepackage{tocloft}
\setlength\cftparskip{0pt}
\setlength\cftbeforesecskip{3pt}
\setlength\cftaftertoctitleskip{6pt}

% Head/Foot
\usepackage{fancyhdr}
\usepackage{extramarks}
\pagestyle{fancy}
\fancyhf{} % alle Kopf- und Fusszeilenfelder bereinigen
\fancyhead[L]{\textsc{\firstleftmark}}
\fancyhead[C]{}
\fancyhead[R]{\slshape\rightmark}
\renewcommand\sectionmark[1]{
	\markboth{\thesection\ #1}{}
}
\fancyfoot[C]{\thepage} % Seitenzahl
%\renewcommand{\footrulewidth}{0.4pt}
\renewcommand\headrule{{
	\color{black}
	\hrule height 2pt width\headwidth
	\vspace{1pt}
	\hrule height 1pt width\headwidth
	\vspace{-4pt}
}}

\DeclareCaptionFont{CaptionFontSize}{\fontsize{10pt}{12pt}\selectfont}
\captionsetup{font=CaptionFontSize}
		

\newcommand{\code}[1]{\texttt{#1}}

% Code
\usepackage{listings}  % Darstellung von Code mit Syntaxhighlighting
\usepackage{setspace}
\usepackage{color}

\DeclareFixedFont{\ttb}{T1}{txtt}{bx}{n}{8} % for bold
\DeclareFixedFont{\ttm}{T1}{txtt}{m}{n}{8}  % for normal

\lstset{
  backgroundcolor=\color{white},   % choose the background color; you must add \usepackage{color} or \usepackage{xcolor}; should come as last argument
  basicstyle=\ttm,        % the size of the fonts that are used for the code
  breakatwhitespace=false,         % sets if automatic breaks should only happen at whitespace
  breaklines=true,                 % sets automatic line breaking
  captionpos=b,                    % sets the caption-position to bottom
  commentstyle=\color{mygreen},    % comment style
  deletekeywords={...},            % if you want to delete keywords from the given language
  escapeinside={\%*}{*)},          % if you want to add LaTeX within your code
  extendedchars=true,              % lets you use non-ASCII characters; for 8-bits encodings only, does not work with UTF-8
  firstnumber=1,                % start line enumeration with line 1000
  frame=tb,	                   % adds a frame around the code
  keepspaces=true,                 % keeps spaces in text, useful for keeping indentation of code (possibly needs columns=flexible)
  keywordstyle=\color{blue},       % keyword style
  language=Python,                 % the language of the code
  morekeywords={*,...},            % if you want to add more keywords to the set
  otherkeywords={self, None, True, False, with},  
  numbers=left,                    % where to put the line-numbers; possible values are (none, left, right)
  numbersep=5pt,                   % how far the line-numbers are from the code
  numberstyle=\tiny\color{mygray}, % the style that is used for the line-numbers
  rulecolor=\color{black},         % if not set, the frame-color may be changed on line-breaks within not-black text (e.g. comments (green here))
  showspaces=false,                % show spaces everywhere adding particular underscores; it overrides 'showstringspaces'
  showstringspaces=false,          % underline spaces within strings only
  showtabs=false,                  % show tabs within strings adding particular underscores
  stepnumber=1,                    % the step between two line-numbers. If it's 1, each line will be numbered
  stringstyle=\color{mymauve},     % string literal style
  tabsize=2,	                   % sets default tabsize to 2 spaces
  title=\lstname,                   % show the filename of files included with \lstinputlisting; also try caption instead of title
  emph={__init__, @staticmethod},          % Custom highlighting
  emphstyle=\ttb\tiny\color{deepred},    % Custom highlighting style
  stringstyle=\color{mygreen},
}


% Syntaxhighlighting for XML-code
% Source: https://tex.stackexchange.com/questions/10255/xml-syntax-highlighting
\definecolor{gray}{rgb}{0.4,0.4,0.4}
\definecolor{darkblue}{rgb}{0.0,0.0,0.6}
\definecolor{cyan}{rgb}{0.0,0.6,0.6}

\lstdefinelanguage{myXML}{
  morestring=[b]",
  morestring=[s]{>}{<},
  morecomment=[s]{<?}{?>},
  stringstyle=\color{black},
  identifierstyle=\color{darkblue},
  keywordstyle=\color{cyan},
  morekeywords={xmlns,version,type}
}

\lstset{
  literate={ö}{{\"o}}1
           {ä}{{\"a}}1
           {ü}{{\"u}}1
}

% Links
\PassOptionsToPackage{hyphens}{url}
\usepackage{hyperref}
\hypersetup{
    pdftoolbar=true,
    pdfmenubar=true,
    pdffitwindow=false,
    pdfstartview={FitH},
    pdftitle={title},
    pdfauthor={Bernard Schroffenegger},
    pdfsubject={Korpusversionen},
    pdfcreator={Bernard Schroffenegger},
    pdfproducer={Bernard Schroffenegger},
    pdfkeywords={Korpus} {Korpora} {Vergleich} {Versionen} {Werkzeug},
    pdfnewwindow=true,
    colorlinks,
    linkcolor={blue!50!black},
    citecolor={blue!50!black},
    urlcolor={blue!80!black},
	filecolor=magenta
}


% \usepackage{breakurl}

% Fussnoten
\renewcommand{\thefootnote}{\arabic{footnote}}

% Symbol über und unter "=":
% https://tex.stackexchange.com/questions/123219/writing-above-and-below-a-symbol-simultaneously, 6.5.2019
\usepackage{stackengine}
\newcommand\stackequal[2]{%
  \mathrel{\stackunder[2pt]{\stackon[4pt]{$\Rightarrow$}{$\scriptscriptstyle#1$}}{%
  $\scriptscriptstyle#2$}}}

% Tabellen
\usepackage{multirow}

\tikzstyle{bag} = [align=center]

% Bilder
\usepackage{caption}
\usepackage{subcaption}

% Tabellen
\usepackage{booktabs}  % toprule, midrule, bottomrule, \cmidrule{...}, ...

\usepackage{caption}
\usepackage{subcaption}


\begin{document}
\sloppy  % Silbentrennung+


\thispagestyle{empty}  % weder Head/Foot noch Seitenzahl
\pagenumbering{gobble}  % anonyme Seite (vom Seitenzähler ignoriert)




\vspace*{-2cm}
\noindent\rule{1\textwidth}{0.5pt}

\vspace*{0.2cm}
\begin{minipage}[t]{0.4\textwidth}
\begin{flushleft}
\hspace*{-0.4cm}\includegraphics[width=1\textwidth]{Bilder/UZH_logo.png}
\end{flushleft}
\end{minipage}
\begin{minipage}[t]{0.54\textwidth}
\begin{flushright}
Institut für Computerlinguistik\\
Institut für Schweizerische Reformationsgeschichte
\end{flushright}
\end{minipage}

\vspace*{0.3cm}
\noindent\rule{1\textwidth}{0.5pt}\\


\begin{center}
\vspace*{0.5cm}
\textbf{{\Huge\textsc{
Crowd Correction Initiative\\
zur Digitalisierung von\\
Bullingers Briefwechsel\\
}}}

\rule{0.72\textwidth}{3pt}\\
\begin{Large}

Projektdokumentation

%\vfill
%\begin{center}
%\includegraphics[scale=0.8]{Bilder/Bullinger.jpg}
%\end{center}
\vfill

%\begin{footnotesize}
%von\\
%Bernard Schroffenegger\\
%\url{bernard.schroffenegger@uzh.ch}
%\end{footnotesize}

\vspace*{1cm}
Version 1

\medskip
\today

\end{Large}
\end{center}


%===================
% Inhaltsverzeichnis
\newpage
\tableofcontents
\newpage


\section{Anforderungsspezifikation}


\subsection{IST-Zustand}


\subsubsection{Schema der Karteikarten}

{\small \begin{figure}[H]
\centering
	\begin{subfigure}[c]{0.47\textwidth}
		\centering
		\subcaption{\code{Karteikarten\_HBBW\_1551\_100, S.13/99}}
		\includegraphics[scale=0.25]{Bilder/Karteikarte_Beispiel.png}
	\end{subfigure}
	\begin{subfigure}[c]{0.47\textwidth}
		\centering
		\subcaption{\code{Karteikarten\_HBBW\_1551\_100, S.5/99}}
		\includegraphics[scale=0.38]{Bilder/Exception.png}
	\end{subfigure}
\caption{typische Karteikarte (links) und Spezialfall (rechts)}
\end{figure}

}

\vspace*{-0.3cm}
{\small \begin{figure}[H]
\centering
\begin{footnotesize}
\begin{tabular}{ccl}
\toprule
\# Felder & \# Attribute & Attribute\\
\midrule
$7\times$	&	1 & Datum; Absender; Empfänger; Sprache; Literatur; Gedruckt; Bemerkung \\
$2\times$	&	2 & [Photokopie, Bull. Corr.]; [Abschrift, Bull. Corr.]\\
$2\times$	& 	4 & [Autograph, Standort, Sign., Umfang]; [Kopie, Standort, Sign. Umfang]\\
\midrule
$11\times$	&	19 & $\sum$\\
\bottomrule
\end{tabular}
\end{footnotesize}
\caption{Felder und Attribute}
\end{figure}}

Die Attributnamen \quotes{\code{Standort}}, \quotes{\code{Sign.}}, \quotes{\code{Umfang}} und \quotes{\code{Bull. Corr.}} sind auf den Karteikarten doppelt enthalten.

\bigskip

\subsubsection{Schemata nach OCR}

\begin{itemize}
\item Version 1:
\url{http://www.abbyy.com/FineReader_xml/FineReader10-schema-v1.xml}

\item Version 2:
\url{http://www.loc.gov/standards/alto/alto-v2.0.xsd}
\end{itemize}

Positionsangaben:


\begin{figure}[H]
\begin{minipage}[ct]{0.47\textwidth}

\begin{center}
\begin{tikzpicture}[fill=blue!20, scale=0.4]

% Koordinaten/Beschriftungen
\draw[line width=2pt] (0,-5) -- (0,0) -- (10,0);
\draw (-2,0) node[anchor=south]{\scriptsize (\code{page})};
%\draw (0,-0.6) node[anchor=east]{\scriptsize \code{page} (links)};

% Element
\fill (2,-2) -- (6,-2) -- (6,-3) -- (2,-3) -- (2,-2);

% Pfeile horizontal
\draw[>=latex,<->] (0,-3.5) -- (2, -3.5);
\draw[>=latex,<->] (0,-3.8) -- (6, -3.8);
\draw (1, -3.5) node[anchor=south]{{\scriptsize $l$}};
\draw (1, -3.8) node[anchor=north]{{\scriptsize $r$}};

% Pfeile vertikal
\draw[>=latex,<->] (6.5,0) -- (6.5, -2);
\draw[>=latex,<->] (6.8,0) -- (6.8, -3);
\draw (6.5, -1) node[anchor=east]{{\scriptsize $t$}};
\draw (6.8, -1.5) node[anchor=west]{{\scriptsize $b$}};

% Hilfslinien
\draw[red, densely dotted] (1.7,-2) -- (6.6,-2);
\draw[red, densely dotted] (1.7,-3) -- (7.1,-3);
\draw[red, densely dotted] (2,-1.7) -- (2,-3.7);
\draw[red, densely dotted] (6,-1.7) -- (6,-4);

% Diag
\draw (4,-2.5) circle (2pt);
\draw[dotted, line width=0.5pt] (2,-2) -- (6,-3);
\draw[dotted, line width=0.5pt] (2,-3) -- (6,-2);
\draw (4,-2.5) node[anchor=north]{{\footnotesize $S$}};

% Koordinaten
\draw[dashed] (4,-2.5) -- (4,0) node[anchor=south]{$S_x$};
\draw[dashed] (4,-2.5) -- (0,-2.5) node[anchor=east]{$S_y$};
\end{tikzpicture}

		
\begin{eqnarray*}
	S_{xy}	&=& \left(\frac{r+l}{2}, \frac{t+b}{2}\right)
\end{eqnarray*}		

\end{center}
\end{minipage}
\begin{minipage}[ct]{0.47\textwidth}

\begin{center}
\begin{tikzpicture}[fill=blue!20, scale=0.4]
% Koordinaten/Beschriftungen
\draw[line width=2pt] (0,-5) -- (0,0) -- (10,0);
% \draw (-2,0) node[anchor=south]{\scriptsize \code{page}};

% Element
\fill (2,-2) -- (6,-2) -- (6,-3) -- (2,-3) -- (2,-2);

\draw[>=latex,<->] (2,0) -- (2,-2);
\draw (2,-1) node[anchor=west]{{\tiny VPOS}};
\draw[>=latex,<->] (0,-2) -- (2,-2);
\draw (1,-2) node[anchor=south]{{\tiny HPOS}};
\draw[>=latex,<->] (2,-3.3) -- (6,-3.3);
\draw (4,-3.3) node[anchor=north]{{\tiny WIDTH}};
\draw[>=latex,<->] (6.3,-2) -- (6.3,-3);
\draw (6.3,-2.5) node[anchor=west]{{\tiny HEIGH}};

\draw[red, densely dotted] (2,-3)--(2,-3.5);
\draw[red, densely dotted] (6,-3)--(6,-3.5);
\draw[red, densely dotted] (6,-3)--(6.5,-3);
\draw[red, densely dotted] (6,-2)--(6.5,-2);

\draw[dashed] (4,0)--(4,-2.5);
\draw[dashed] (0,-2.5)--(4,-2.5);

\draw (4,0) circle (2pt);
\draw (4,-2.5) circle (2pt);
\draw (0,-2.5) circle (2pt);

\draw (4,0) node[anchor=south]{$S_x$};
\draw (4,-2.5) node[anchor=west]{$S$};
\draw (0,-2.5) node[anchor=east]{$S_y$};

\end{tikzpicture}

\begin{eqnarray*}
S_{xy} = \left(\text{HPOS}+\frac{\text{WIDTH}}{2}, \text{VPOS}+\frac{\text{HEIGHT}}{2}\right)
\end{eqnarray*}	
\end{center}

\end{minipage}
\caption{Schwerpunktskoordinaten $S_{xy}$ von Elementen in Version 1 (links) und 2.}
\label{fig:Schwerpunkte}
\end{figure}




%
%
%\ssss{Struktur Version 1}
%
%\noindent
%Schema: \url{http://www.abbyy.com/FineReader_xml/FineReader10-schema-v1.xml}
%
%\medskip\noindent
%Beobachtungen am Beispiel der Datei \code{Karteikarten\_HBBW\_1551\_1000012.xml}:
%
%\begin{itemize}
%
%\item
%Das \code{<page>}-Element beschreit den Inhalt der Karteikarte (9851x6994px) und gliedert sich zunächst seriell in verschiedene \code{<block>}-Elemente. Grösse und Position derselben sind durch die \code{t,r,b,l}-Attribute (top, right, bottom, left) exakt definiert, welche sich stets auf die linke/obere Seite der Karteikarte (\code{<page>}) beziehen.
%
%\begin{figure}[H]
\begin{minipage}[c]{0.47\textwidth}
	\begin{flushright}
		
\begin{tikzpicture}[fill=blue!20, scale=0.4]

% Koordinaten/Beschriftungen
\draw[line width=2pt] (0,-5) -- (0,0) -- (10,0);
\draw (-2,0) node[anchor=south]{\scriptsize obere Seitenkante (\code{page})};
\draw (0,-0.6) node[anchor=east]{\scriptsize \code{page} (links)};

% Element
\fill (2,-2) -- (6,-2) -- (6,-3) -- (2,-3) -- (2,-2);

% Pfeile horizontal
\draw[>=latex,<->] (0,-3.5) -- (2, -3.5);
\draw[>=latex,<->] (0,-3.8) -- (6, -3.8);
\draw (1, -3.5) node[anchor=south]{{\scriptsize $l$}};
\draw (1, -3.8) node[anchor=north]{{\scriptsize $r$}};

% Pfeile vertikal
\draw[>=latex,<->] (6.5,0) -- (6.5, -2);
\draw[>=latex,<->] (6.8,0) -- (6.8, -3);
\draw (6.5, -1) node[anchor=east]{{\scriptsize $t$}};
\draw (6.8, -1.5) node[anchor=west]{{\scriptsize $b$}};

% Hilfslinien
\draw[red, densely dotted] (1.7,-2) -- (6.6,-2);
\draw[red, densely dotted] (1.7,-3) -- (7.1,-3);
\draw[red, densely dotted] (2,-1.7) -- (2,-3.7);
\draw[red, densely dotted] (6,-1.7) -- (6,-4);

% Diag
\draw (4,-2.5) circle (2pt);
\draw[dotted, line width=0.5pt] (2,-2) -- (6,-3);
\draw[dotted, line width=0.5pt] (2,-3) -- (6,-2);
\draw (4,-2.5) node[anchor=north]{{\footnotesize $S$}};

% Koordinaten
\draw[dashed] (4,-2.5) -- (4,0) node[anchor=south]{$S_x$};
\draw[dashed] (4,-2.5) -- (0,-2.5) node[anchor=east]{$S_y$};
\end{tikzpicture}

	\end{flushright}
\end{minipage}
\begin{minipage}[c]{0.47\textwidth}
\begin{flushleft}
		\begin{eqnarray*}
			S_{xy}	&=& \left(\frac{r+l}{2}, \frac{t+b}{2}\right)
		\end{eqnarray*}
\end{flushleft}
\end{minipage}
\caption{Koordinaten des Schwerpunkts $S_{xy}$ eines graphischen Elements}
\label{fig:mittelpunkt}
\end{figure}
%
%\item
%Das erste \code{<block>}-Element enthält eine Tabelle (\code{blockType="'Table"'}) mit den OCR-Daten (\code{<row>}/\code{<cell>}). Die Höhe der einzelnen Zeilen ist nur implizit durch deren Inhalt gegeben (Zellen, \code{height}/\code{width}). Alle weiteren \code{<block>}-Elemente beschreiben Linien oder Rechtecke (\code{blockType="'Separator"'}).
%
%\vspace*{-0.7cm}
%\begin{figure}[H]
%\begin{eqnarray*}
%\mu_{\text{(\#rows)}}&\pm &\sigma_{\text{(\#rows)}} = 8.09\pm 2.32\\
%\mu_{\text{(\#cells)}}&\pm &\sigma_{\text{(\#cells)}} = 21.96\pm 10.38\\
%\mu_{\text{(\#texts)}}&\pm &\sigma_{\text{(\#texts)}} = 21.29\pm 10.43\\
%\end{eqnarray*}
%
%\vspace*{-1cm}
%\begin{quote}
%\begin{quote}
%\caption{Anzahl Zeilen/Zellen variieren von Karteikarte zu Karteikarte: Alignierte Textelemente ergeben jeweils \textit{potentiell} eine Zeile.\\
%Code: \hyperref[code:xml_analysen]{\code{xml.py/calculate\_element\_stats(directory\_path\_to\_xml\_files)}}\\
%Alle Daten: siehe \autoref{tabelle:ocr_statistik}}
%\end{quote}
%\end{quote}
%\end{figure}
%
%\vspace*{-1.3cm}
%\item
%Die Höhe/Breite einzelner Zellen einer Zeile sind explizit gegeben (\code{width}/\code{height}).
%
%\item
%Zellen enthalten in \code{<par>}-Elementen gegliederten Text (\code{<text>}). Paragraphen definieren den Zeilenabstand (\code{lineSpacing}) der darin enthaltenen Textzeilen (\code{<line>}), wie auch eine Einrückung (\code{leftIndent}/\code{align}).
%
%\item
%Das \code{baseline}-Attribute einer Linie definiert den Abstand zwischen dem oberen Rand der Seite (\code{<page>}) und dem Text (den Zeichendaten) innerhalb einer Linie (\code{<formatting>}).
%
%\item
%Die Grösse/Position einer Textbox (\code{<line>}/\code{<formatting>}) ist durch \code{t,r,b,l}-Attribute gegeben, die sich (wie die \code{block}-Elemente) auf die Ränder der gesamten Seite beziehen (siehe Abbildung \ref{fig:mittelpunkt}, links).
%\end{itemize}
%
%
%
%\ssss{Struktur Version 2}
%
%\noindent
%Schema: \url{http://www.loc.gov/standards/alto/alto-v2.0.xsd}
%
%\medskip\noindent
%Beobachtungen am Beispiel der Datei \code{Karteikarten\_HBBW\_1551\_1000012.xml}:
%
%\begin{itemize}
%
%\item
%Die Daten einer Karteikarte befinden sich innerhalb des \code{<page>}- (\code{height}/\code{width}), bzw. des darin enthaltenen \code{<PrintSpace>}-Elements (\code{height, width, vpos, hpos}). Dieses beinhaltet  \code{<GraphicalElement>}-Elemente und (genau) ein \code{<ComposedBlock>}-Element (\code{height, width, vpos, hpos}).
%
%\item
%Das \code{<ComposedBlock>}-Element der Karteikarte glieder sich in \code{<TextBlock>}-Elemente, welche ein oder mehrere \quotes{Textzeilen} (\code{<TextLine>}) enthalten (\code{height-, width-, vpos-, hpos-} sowie ein \code{baseline-}Attribut).
%
%\item
%Letztere enthalten u.A. \code{<String>}-Elemente (\code{CONTENT}, \code{HEIGHT}, \code{WIDTH}, \code{VPOS} \& \code{HPOS}). Das \code{CONTENT}-Attribut enthält jeweils \quotes{ein Stück} Zeichendaten, und die anderen Attribute definieren dessen Position.
%
%
%\begin{figure}[H]
%\begin{minipage}[c]{0.47\textwidth}
%	\begin{flushright}
%\begin{tikzpicture}[fill=blue!20, scale=0.4]
%% Koordinaten/Beschriftungen
%\draw[line width=2pt] (0,-5) -- (0,0) -- (10,0);
%\draw (-2,0) node[anchor=south]{\scriptsize obere Seitenkante (\code{page})};
%
%% Element
%\fill (2,-2) -- (6,-2) -- (6,-3) -- (2,-3) -- (2,-2);
%
%\draw[>=latex,<->] (2,0) -- (2,-2);
%\draw (2,-1) node[anchor=west]{{\tiny VPOS}};
%\draw[>=latex,<->] (0,-2) -- (2,-2);
%\draw (1,-2) node[anchor=south]{{\tiny HPOS}};
%\draw[>=latex,<->] (2,-3.3) -- (6,-3.3);
%\draw (4,-3.3) node[anchor=north]{{\tiny WIDTH}};
%\draw[>=latex,<->] (6.3,-2) -- (6.3,-3);
%\draw (6.3,-2.5) node[anchor=west]{{\tiny HEIGH}};
%
%\draw[red, densely dotted] (2,-3)--(2,-3.5);
%\draw[red, densely dotted] (6,-3)--(6,-3.5);
%\draw[red, densely dotted] (6,-3)--(6.5,-3);
%\draw[red, densely dotted] (6,-2)--(6.5,-2);
%
%\draw[dashed] (4,0)--(4,-2.5);
%\draw[dashed] (0,-2.5)--(4,-2.5);
%
%\draw (4,0) circle (2pt);
%\draw (4,-2.5) circle (2pt);
%\draw (0,-2.5) circle (2pt);
%
%\draw (4,0) node[anchor=south]{$S_x$};
%\draw (4,-2.5) node[anchor=west]{$S$};
%\draw (0,-2.5) node[anchor=east]{$S_y$};
%
%\end{tikzpicture}
%	\end{flushright}
%\end{minipage}
%\begin{minipage}[c]{0.47\textwidth}
%\begin{flushleft}
%{\footnotesize $S_{xy} = \left(\text{HPOS}+\frac{\text{WIDTH}}{2}, \text{VPOS}+\frac{\text{HEIGHT}}{2}\right)$}
%\end{flushleft}
%\end{minipage}
%\caption{Koordinaten des Schwerpunkts $S_{xy}$ eines graphischen Elements}
%\label{fig:mittelpunkt}
%\end{figure}
%
%
%
%
%
%\end{itemize}







\subsection{SOLL-Zustand}

\subsubsection{Schema der Daten}

Wir sollten die Daten so differenziert wie möglich erfassen (Datum $\rightarrow$ [Jahr, Monat, Tag]), Redundantes entfernen (Bull. Corr. $\rightarrow$ [Bull. Corr., Blatt, Seite]), und Attributwerte normieren (Jan., 1., 01., etc. $\rightarrow$ Januar).

{\small \begin{figure}[H]
\centering
	\begin{subfigure}[c]{0.47\textwidth}
		\centering
		\subcaption{\code{Karteikarten\_HBBW\_1551\_100, S.13/99}}
		\includegraphics[scale=0.25]{Bilder/Karteikarte_Beispiel.png}
	\end{subfigure}
	\begin{subfigure}[c]{0.47\textwidth}
		\centering
		\subcaption{\code{Karteikarten\_HBBW\_1551\_100, S.5/99}}
		\includegraphics[scale=0.38]{Bilder/Exception.png}
	\end{subfigure}
\caption{typische Karteikarte (links) und Spezialfall (rechts)}
\end{figure}

}





\subsection{Anforderungsanalyse}


\subsubsection{Anwendungsfälle (User Stories)}

Die folgenden Anwendungsszenarien dienen zur Analyse der Softwareanforderungen und so als Basis für die Formulierung der funktionalen Anforderungen. 

Als Besucher/Anwender der Website/-applikation möchte ich...

\begin{itemize}

\item die Webseite über einen Link erreichen,
\item allgemeine Informationen über Sinn/Zweck der Initiative erhalten,
\item 

\end{itemize}







\subsubsection{Funktionale Anforderungen}

Webapplikation 



\ssss{Front-End (Client: Webbrowser)}

\begin{itemize}

\item Erreichbare Website
\item Benutzer Authentifizierung (Anmeldung)
\item 

\end{itemize}



\ssss{Back-End} (Server)

\begin{itemize}

\item

\end{itemize}




\subsection{Turkle}

\subsubsection{Dokumentation}
Offiziell

\begin{itemize}
\item Allgemein
\url{https://github.com/hltcoe/turkle}

\item Server/DB
\url{https://github.com/hltcoe/turkle/blob/master/docs/ADMINISTRATION.md}

\item Templates
\url{https://github.com/hltcoe/turkle/blob/master/docs/TEMPLATE-GUIDE.md}
\end{itemize}



\begin{itemize}
\item
Download Turkle from
\url{https://github.com/hltcoe/turkle.git}

\item
Dependencies:
\code{pip install -r requirements.txt}

\item
Database (Init):
\code{python manage.py migrate}

\item
Admin:
\code{python manage.py createsuperuser}

\item
Upgrade.
\code{python manage.py migrate}

\item
Start Server
\code{python manage.py runserver 0.0.0.0:8000}

\end{itemize}














\newpage
\section{Implementation}

\subsection{Datenextraktion}

\subsubsection{Karteikartengrösse}

Die Seitengrössen beider OCR-Versionen stimmen exakt überein.

\begin{center}
\begin{minipage}[t]{0.45\textwidth}
\begin{center}
\includegraphics[scale=0.45]{Bilder/page_size.png}
\end{center}
\end{minipage}
\begin{minipage}[t]{0.45\textwidth}
\begin{center}
\includegraphics[scale=0.45]{Bilder/page_size_adjusted.png}
\end{center}
\end{minipage}
\end{center}

\vspace*{-1.5cm}

\begin{center}
\begin{minipage}[t]{0.45\textwidth}

\begin{center}
\begin{table}[H]
\centering
\begin{scriptsize}
\begin{tabular}{lcccc}
\toprule
Seite &  $\mu_a$ &  $\sigma_a$ &  min &  max \\
\midrule
Breite	&    9661  &		975 &     4922 &     9902 \\
Höhe 	&    6837  &		690 &     3488 &     7012 \\
\bottomrule
\end{tabular}
\caption{Dimensionen einer Karteikarte}
\end{scriptsize}
\end{table}
\end{center}

\end{minipage}
\begin{minipage}[t]{0.45\textwidth}

\begin{center}
\begin{table}[H]
\centering
\begin{scriptsize}
\begin{tabular}{lcccc}
\toprule
Seite &  $\mu_b$ &  $\sigma_b$ &  min &  max \\
\midrule
Breite	&    \textbf{9860} &		11 & 9843 & 9902 \\
Höhe	&    \textbf{6978} &		10 & 6949 & 7012 \\
\bottomrule
\end{tabular}
\caption{ohne Ausreisser}
\end{scriptsize}
\end{table}
\end{center}

\end{minipage}
\end{center}

\vspace*{-0.3cm}
Durch Ausreisser verursachter Fehler:


\begin{footnotesize}
\begin{eqnarray*}
\Delta\mu_{x}^\% &=& 1 - \frac{9661}{9860} = 2.0183 \%\\
\Delta\mu_{y}^\% &=& 1 - \frac{6837}{6978} = 2.0206 \%
\end{eqnarray*}
\end{footnotesize}


\vspace*{-0cm}
Ausreisser: $4/99 \approx 4 \%$

\begin{scriptsize}
\begin{itemize}
\setlength\itemsep{-0.1em}
\item $(4922, 3488)$ \code{Karteikarten\_HBBW\_1551\_1000010.xml}
\item $(4923, 3489)$ \code{Karteikarten\_HBBW\_1551\_1000041.xml}
\item $(4935, 3492)$ \code{Karteikarten\_HBBW\_1551\_1000069.xml}
\item $(4937, 3489)$ \code{Karteikarten\_HBBW\_1551\_1000095.xml}
\end{itemize}
\end{scriptsize}



Skalierung für Seitenlängen $(x,y)^T < \vec{\mu_b} - 4\vec{\sigma_b}$:

\begin{eqnarray}
\mu_{bx}	&=& \alpha\mu_{ax}
\quad\Leftrightarrow\quad \alpha = \frac{\mu_{bx}}{\mu_{ax}}
\quad\Rightarrow\quad\alpha (\mu_{ax}) = \frac{9860}{\mu_{ax}}\\
\mu_{by}	&=& \beta\mu_{ay}
\quad\Leftrightarrow\quad \beta = \frac{\mu_{by}}{\mu_{ay}}
\quad\Rightarrow\quad\beta (\mu_{ay}) = \frac{6978}{\mu_{ay}}
\end{eqnarray}








\subsubsection{Datenfelder}
\vspace*{0cm}

% Bilder: Manuelle Vermessung/Partitionierung
\begin{figure}[H]
	\begin{minipage}[t]{0.45\textwidth}
		\begin{center}
			\includegraphics[scale=0.25]{Bilder/Karteikarte_Original.png}
		\end{center}
	\end{minipage}
	\begin{minipage}[t]{0.45\textwidth}
		\begin{center}
			\includegraphics[scale=0.25]{Bilder/Karteikartenmasse.png}
		\end{center}
	\end{minipage}
    \caption{Manuelle Vermessung/Partitionierung einer Karteikarte}
    \label{fig:sample_figure}
\end{figure}

% Berechnung
\begin{lstlisting}
f = lambda l: [sum(l[:i+1]) for i, _ in enumerate(l)]  # Partialsummen
f([b*9860 for b in [0.31, 0.35, 0.34]])                # [3057, 6508, 9860]
f([h*6978 for h in [0.22, 0.29, 0.1, 0.39]])           # [1535, 3559, 4257, 6978]
f([h*6978 for h in [0.22, 0.14, 0.15, 0.25, 0.24]])    # [1535, 2512, 3559, 5303, 6978]
[1535+i*0.29*6978/4 for i in range(1,5)]               # [2041, 2547, 3053, 3559]
\end{lstlisting}





\subsubsection{OCR-Text}
\vspace*{-0.3cm}

% Schwerpunkte
\begin{figure}[H]
\centering
	\begin{subfigure}[t]{0.47\textwidth}
	\centering
		\subcaption{\code{ocr\_sample\_100\_v1}}		
		\includegraphics[scale=0.5]{Bilder/scatter_v1_final.png}
	\end{subfigure}
	\begin{subfigure}[t]{0.47\textwidth}
	\centering
		\subcaption{\code{ocr\_sample\_100\_v2}}		
		\includegraphics[scale=0.5]{Bilder/scatter_v2_final.png}
	\end{subfigure}
\caption{Schwerpunkte von OCR-Text-Elementen von jeweils 100 Karteikarten}
\end{figure}


\subsubsection{OCR-Attribute}
\vspace*{-0.3cm}

Attribute total: 1728. Gefiltert (FP): 1660

\begin{figure}[H]
\centering
	\begin{subfigure}[t]{0.243\textwidth}
	\centering
		%\subcaption{\code{ocr\_sample\_100\_v1}}		
		\includegraphics[scale=0.28]{Bilder/attributes_0.png}
	\end{subfigure}
	\begin{subfigure}[t]{0.243\textwidth}
	\centering
		%\subcaption{\code{ocr\_sample\_100\_v2}}		
		\includegraphics[scale=0.28]{Bilder/attributes_1.png}
	\end{subfigure}
	\begin{subfigure}[t]{0.243\textwidth}
	\centering
		%\subcaption{\code{ocr\_sample\_100\_v1}}		
		\includegraphics[scale=0.28]{Bilder/attributes_2.png}
	\end{subfigure}
	\begin{subfigure}[t]{0.243\textwidth}
	\centering
		%\subcaption{\code{ocr\_sample\_100\_v2}}		
		\includegraphics[scale=0.28]{Bilder/attributes_3.png}
	\end{subfigure}
\caption{Verteilung einzelner Attributnamen}
\end{figure}





% \ssss{Häufungen (Koordinaten)}

\vspace*{-0.6cm}
\begin{figure}[H]
\centering
	\begin{subfigure}[t]{0.47\textwidth}
	\begin{flushright}
		%\centering\subcaption{\code{ocr\_sample\_100\_v1}}
		\includegraphics[scale=0.5]{Bilder/x_distribution_v1.png}
	\end{flushright}
	\end{subfigure}
	\begin{subfigure}[t]{0.47\textwidth}
	\begin{flushleft}
		%\centering\subcaption{\code{ocr\_sample\_100\_v2}}
		\includegraphics[scale=0.5]{Bilder/x_distribution_v2.png}
	\end{flushleft}
	\end{subfigure}
\caption{Häufigkeiten verschiedener $x$-Koordinaten}
\end{figure}


\vspace*{-0.6cm}
\begin{figure}[H]
\centering
	\begin{subfigure}[t]{0.47\textwidth}
	\begin{flushright}
	%\centering\subcaption{\code{ocr\_sample\_100\_v1}}
		\includegraphics[scale=0.5]{Bilder/y_distribution_v1.png}
	\end{flushright}	\end{subfigure}
	\begin{subfigure}[t]{0.47\textwidth}
\begin{flushleft}
	%\centering\subcaption{\code{ocr\_sample\_100\_v2}}
		\includegraphics[scale=0.5]{Bilder/y_distribution_v2.png}
	\end{flushleft}
	\end{subfigure}
\caption{Häufigkeiten verschiedener $y$-Koordinaten}
\end{figure}
%
%\ssss{Lücken (Koordinaten)}
%
%Wir teilen eine Koordinatenachse sukzessive in immer kleinere \quotes{Buckets} ein: Die ersten auftretenden leeren Buckets sollten erwartungsgemäss mit den Feldgrenzen der Attribute korrelieren.
%
%\begin{scriptsize}
%\begin{table}[H]
%\begin{center}
%\begin{minipage}[t]{0.45\textwidth}
%\centering
%$x$-Koordinaten\\
%\smallskip
%\includegraphics[scale=0.5]{Bilder/gaps_x.png}
%\end{minipage}
%\begin{minipage}[t]{0.45\textwidth}
%\centering
%$y$-Koordinaten\\
%\smallskip
%\includegraphics[scale=0.5]{Bilder/gaps_y.png}
%\end{minipage}
%\end{center}
%\vspace*{-0.5cm}
%\caption{Abschätzung der Feldgrenzen}
%\end{table}
%\end{scriptsize}

\begin{figure}[H]
\centering
	\begin{subfigure}[t]{0.47\textwidth}
	\begin{flushright}
		%\centering\subcaption{\code{ocr\_sample\_100\_v1}}
		\includegraphics[scale=0.5]{Bilder/ocr_attributes_1.png}
	\end{flushright}
	\end{subfigure}
	\begin{subfigure}[t]{0.47\textwidth}
	\begin{flushleft}
		%\centering\subcaption{\code{ocr\_sample\_100\_v2}}
		\includegraphics[scale=0.5]{Bilder/ocr_attributes_2.png}
	\end{flushleft}
	\end{subfigure}
\caption{Durchschnittliche Attributpositionen (korrigiert/unkorrigiert)} 
\end{figure}

Lineare Separierung:

\begin{minipage}{0.45\textwidth}
\begin{center}
\begin{eqnarray*}
y[\text{Standort}_1] &\approx & y[\text{Standort}_2]\\
y[\text{Sign.}_1] &\approx & y[\text{Sign.}_2]\\
y[\text{Umfang}_1] &\approx & y[\text{Umfang}_2]\\
x[\text{Bull. Corr.}_1] &\approx & x[\text{Bull. Corr.}_2]
\end{eqnarray*}
\end{center}
\end{minipage}
\begin{minipage}{0.45\textwidth}
\begin{eqnarray*}
x[\text{Standort}_1] &<& x[\text{Standort}_2]\\
x[\text{Sign.}_1] &<& x[\text{Sign.}_2]\\
x[\text{Umfang}_1] &<& x[\text{Umfang}_2]\\
y[\text{Bull. Corr.}_1] &< & y[\text{Bull. Corr.}_2]
\end{eqnarray*}
\end{minipage}



\begin{figure}[H]
\centering
	\begin{subfigure}[t]{0.47\textwidth}
	\begin{flushright}
		%\centering\subcaption{\code{ocr\_sample\_100\_v1}}
		\includegraphics[scale=0.5]{Bilder/ocr_attributes_3.png}
	\end{flushright}
	\end{subfigure}
	\begin{subfigure}[t]{0.47\textwidth}
	\begin{flushleft}
		%\centering\subcaption{\code{ocr\_sample\_100\_v2}}

	\end{flushleft}
	\end{subfigure}
\caption{Attributwerte} 
\end{figure}




\subsubsection{Algorithmus}





\subsection{Webserver}

Flask (Framework für Webapplikationen); nutzt \quotes{Werkzeug WSGI} (Web Server Gateway Interface) als universelle Schnittstelle wz. Webservern und -applikationen und Jinja2 und rendern dynamischer Webseiten)). Benötigt mindestens Python 2.7.

\begin{lstlisting}
pip install Flask
\end{lstlisting}


















%\begin{scriptsize}
%\begin{table}[H]
%\begin{center}
%\begin{minipage}[t]{0.45\textwidth}
%\centering
%$x$-Koordinaten\\
%\smallskip
%{\scriptsize \begin{tabular}{lrr}
\toprule
    Attribut &  Mittelwert &  Standardabweichung \\
\midrule
       Datum &     358.368 &              38.837 \\
    Absender &    3554.568 &             367.353 \\
   Empfänger &    6980.688 &             728.437 \\
   Autograph &     506.853 &              48.753 \\
       Kopie &    3394.216 &             347.094 \\
  Photokopie &    6975.531 &             709.279 \\
   Abschrift &    6902.271 &             708.973 \\
     Sprache &     410.558 &              42.600 \\
   Literatur &    3483.720 &             364.234 \\
    Gedruckt &     441.075 &              44.683 \\
 Bemerkungen &    3657.076 &             414.723 \\
    Standort &     436.913 &              20.113 \\
       Sign. &     297.902 &              18.913 \\
      Umfang &     397.109 &              21.013 \\
    Standort &    3577.613 &              21.187 \\
       Sign. &    3442.154 &              18.373 \\
      Umfang &    3543.304 &              18.970 \\
       Bull. &    6862.250 &              33.249 \\
       Corr. &    7262.607 &              33.791 \\
       Bull. &    6875.556 &              41.000 \\
       Corr. &    7276.385 &              39.513 \\
\bottomrule
\end{tabular}}
%\end{minipage}
%\begin{minipage}[t]{0.45\textwidth}
%\centering
%$y$-Koordinaten\\
%\smallskip
%{\scriptsize \begin{tabular}{lrr}
\toprule
    Attribut &  Mittelwert &  Standardabweichung \\
\midrule
       Datum &     314.736 &              32.194 \\
    Absender &     322.726 &              35.649 \\
   Empfänger &     342.172 &              37.650 \\
   Autograph &    1871.884 &             167.743 \\
       Kopie &    1864.433 &             190.518 \\
  Photokopie &    1866.092 &             189.911 \\
   Abschrift &    2829.979 &             290.712 \\
     Sprache &    3856.284 &             344.727 \\
   Literatur &    3820.161 &             398.384 \\
    Gedruckt &    4520.817 &             408.471 \\
 Bemerkungen &    5547.924 &             628.124 \\
    Standort &    2281.870 &              13.554 \\
       Sign. &    2903.780 &              13.724 \\
      Umfang &    3298.087 &              14.095 \\
    Standort &    2283.839 &              13.327 \\
       Sign. &    2903.923 &              13.203 \\
      Umfang &    3299.609 &              13.523 \\
       Bull. &    2283.321 &              14.128 \\
       Corr. &    2284.321 &              13.902 \\
       Bull. &    3285.037 &              19.779 \\
       Corr. &    3285.692 &              19.495 \\
\bottomrule
\end{tabular}}
%\end{minipage}
%\end{center}
%\vspace*{-0.5cm}
%\caption{Abschätzung der Position der Attributnamen auf einer typischen Karteikarte}
%\end{table}
%\end{scriptsize}




\newpage

\appendix
\section{Anhang}




\subsection{Python Code}


\subsubsection{\LaTeX{}-Compiler}

\begin{spacing}{1}
\lstset{basicstyle=\fontsize{6}{7}\selectfont\ttfamily}
\lstinputlisting{compiler.py}
\end{spacing}


\subsubsection{XML-Analysen}
\label{code:xml_analysen}
\begin{spacing}{1}
\lstset{basicstyle=\fontsize{6}{7}\selectfont\ttfamily}
\lstinputlisting{../../Tools/xml.py}
\end{spacing}




\subsection{Screenshots}


\subsubsection{Karteikarte (Original)}

\begin{figure}[H]
\centering
\includegraphics[scale=0.42]{Bilder/Karteikarte_Beispiel.png}
\caption{Sammlung von Karteikarten (Bilder) im \code{pdf}-Format (HBBW\_1551\_100), S. 13/99}
\end{figure}


\subsubsection{Karteikarte (Spezialfall)}
\begin{figure}[H]
\centering
\includegraphics[scale=0.42]{Bilder/Exception.png}
\caption{Sammlung von Karteikarten (Bilder) im \code{pdf}-Format (HBBW\_1551\_100), S. 13/99}
\end{figure}




\subsection{OCR-Output}

\subsubsection{Version 1}

\noindent
Beispiel: \code{Karteikarten\_HBBW\_1551\_1000012.xml}

\noindent
Schema: \url{https://fr7.abbyy.com/FineReader_xml/FineReader10-schema-v1.xml}

\noindent
Formatter: \url{https://www.freeformatter.com/html-formatter.html#ad-output}

\lstset{language=myXML}
\lstset{basicstyle=\fontsize{7}{8}\selectfont\ttfamily}
\input{Code/ocr_v1_formatted.xml}


\subsubsection{Version 2}

\noindent
Beispiel: \code{Karteikarten\_HBBW\_1551\_1000012.xml}

\input{Code/ocr_v2_formatted.xml}



\subsubsection{Element Frequenzen Statistik}
\label{tabelle:ocr_statistik}

{\small \begin{table}[H]
\begin{scriptsize}
\caption{Mittelwert $\mu$ und Standardabweichung $\sigma$ der Elementfrequenzen in den Dateien der Ordner \code{ocr\_sample\_100\_v1} und \code{ocr\_sample\_100\_v2}.}
\begin{minipage}[t]{0.45\textwidth}
\begin{figure}[H]
\centering
\begin{footnotesize}
\begin{tabular}{lll}
\hline\hline
Element & $\mu$ & $\sigma$\\
\hline
document & 1.0 & 0.0 \\
documentData & 1.0 & 0.0 \\
sections & 1.0 & 0.0 \\
section & 1.12 & 0.46 \\
stream & 1.11 & 0.43 \\
mainText & 1.11 & 0.43 \\
elemId & 1.42 & 1.34 \\
page & 1.0 & 0.0 \\
block & 10.0 & 3.9 \\
region & 10.0 & 3.9 \\
rect & 35.81 & 9.43 \\
row & 8.09 & 2.32 \\
cell & 21.96 & 10.38 \\
text & 21.29 & 10.43 \\
par & 30.76 & 6.52 \\
line & 28.88 & 4.35 \\
formatting & 29.56 & 4.74 \\
separator & 7.83 & 0.86 \\
start & 7.83 & 0.86 \\
end & 7.83 & 0.86 \\
\hline\hline
\end{tabular}
\end{footnotesize}
\begin{quote}
\caption{Version 1}
\end{quote}
\end{figure}

\end{minipage}
\begin{minipage}[t]{0.45\textwidth}
\begin{figure}[H]
\centering
\begin{footnotesize}
\begin{tabular}{lll}
\hline\hline
Element & $\mu$ & $\sigma$\\
\hline
alto & 1.0 & 0.0 \\
Description & 1.0 & 0.0 \\
MeasurementUnit & 1.0 & 0.0 \\
OCRProcessing & 1.0 & 0.0 \\
ocrProcessingStep & 1.0 & 0.0 \\
processingDateTime & 1.0 & 0.0 \\
processingSoftware & 1.0 & 0.0 \\
softwareCreator & 1.0 & 0.0 \\
softwareName & 1.0 & 0.0 \\
softwareVersion & 1.0 & 0.0 \\
Styles & 1.0 & 0.0 \\
ParagraphStyle & 2.64 & 1.68 \\
Layout & 1.0 & 0.0 \\
Page & 1.0 & 0.0 \\
PrintSpace & 1.0 & 0.0 \\
ComposedBlock & 1.09 & 0.32 \\
TextBlock & 17.71 & 6.64 \\
TextLine & 28.67 & 4.38 \\
String & 74.27 & 17.98 \\
SP & 45.06 & 13.08 \\
GraphicalElement & 7.78 & 0.93 \\
HYP & 1.07 & 0.26 \\
TopMargin & 1.0 & 0.0 \\
LeftMargin & 1.0 & 0.0 \\
RightMargin & 1.0 & 0.0 \\
BottomMargin & 1.0 & 0.0 \\
Shape & 1.67 & 0.58 \\
Polygon & 1.67 & 0.58 \\
Illustration & Illustration & 0 \\
\hline\hline
\end{tabular}
\end{footnotesize}
\begin{quote}
\caption{Version 2}
\end{quote}
\end{figure}

\end{minipage}
\end{scriptsize}
\end{table}}















\end{document}
