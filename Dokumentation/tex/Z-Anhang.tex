
\appendix
\section{Anhang}




\subsection{Python Code}


\subsubsection{\LaTeX{}-Compiler}

\begin{spacing}{1}
\lstset{basicstyle=\fontsize{6}{7}\selectfont\ttfamily}
\lstinputlisting{compiler.py}
\end{spacing}


\subsubsection{XML-Analysen}
\label{code:xml_analysen}
\begin{spacing}{1}
\lstset{basicstyle=\fontsize{6}{7}\selectfont\ttfamily}
\lstinputlisting{../../Tools/xml.py}
\end{spacing}




\subsection{Screenshots}


\subsubsection{Karteikarte (Original)}

\begin{figure}[H]
\centering
\includegraphics[scale=0.42]{Bilder/Karteikarte_Beispiel.png}
\caption{Sammlung von Karteikarten (Bilder) im \code{pdf}-Format (HBBW\_1551\_100), S. 13/99}
\end{figure}


\subsubsection{Karteikarte (Spezialfall)}
\begin{figure}[H]
\centering
\includegraphics[scale=0.42]{Bilder/Exception.png}
\caption{Sammlung von Karteikarten (Bilder) im \code{pdf}-Format (HBBW\_1551\_100), S. 13/99}
\end{figure}




\subsection{OCR-Output}

\subsubsection{Version 1}

\noindent
Beispiel: \code{Karteikarten\_HBBW\_1551\_1000012.xml}

\noindent
Schema: \url{https://fr7.abbyy.com/FineReader_xml/FineReader10-schema-v1.xml}

\noindent
Formatter: \url{https://www.freeformatter.com/html-formatter.html#ad-output}

\lstset{language=myXML}
\lstset{basicstyle=\fontsize{7}{8}\selectfont\ttfamily}
\input{Code/ocr_v1_formatted.xml}


\subsubsection{Version 2}

\noindent
Beispiel: \code{Karteikarten\_HBBW\_1551\_1000012.xml}

\input{Code/ocr_v2_formatted.xml}



\subsubsection{Element Frequenzen Statistik}
\label{tabelle:ocr_statistik}

{\small \begin{table}[H]
\begin{scriptsize}
\caption{Mittelwert $\mu$ und Standardabweichung $\sigma$ der Elementfrequenzen in den Dateien der Ordner \code{ocr\_sample\_100\_v1} und \code{ocr\_sample\_100\_v2}.}
\begin{minipage}[t]{0.45\textwidth}
\begin{figure}[H]
\centering
\begin{footnotesize}
\begin{tabular}{lll}
\hline\hline
Element & $\mu$ & $\sigma$\\
\hline
document & 1.0 & 0.0 \\
documentData & 1.0 & 0.0 \\
sections & 1.0 & 0.0 \\
section & 1.12 & 0.46 \\
stream & 1.11 & 0.43 \\
mainText & 1.11 & 0.43 \\
elemId & 1.42 & 1.34 \\
page & 1.0 & 0.0 \\
block & 10.0 & 3.9 \\
region & 10.0 & 3.9 \\
rect & 35.81 & 9.43 \\
row & 8.09 & 2.32 \\
cell & 21.96 & 10.38 \\
text & 21.29 & 10.43 \\
par & 30.76 & 6.52 \\
line & 28.88 & 4.35 \\
formatting & 29.56 & 4.74 \\
separator & 7.83 & 0.86 \\
start & 7.83 & 0.86 \\
end & 7.83 & 0.86 \\
\hline\hline
\end{tabular}
\end{footnotesize}
\begin{quote}
\caption{Version 1}
\end{quote}
\end{figure}

\end{minipage}
\begin{minipage}[t]{0.45\textwidth}
\begin{figure}[H]
\centering
\begin{footnotesize}
\begin{tabular}{lll}
\hline\hline
Element & $\mu$ & $\sigma$\\
\hline
alto & 1.0 & 0.0 \\
Description & 1.0 & 0.0 \\
MeasurementUnit & 1.0 & 0.0 \\
OCRProcessing & 1.0 & 0.0 \\
ocrProcessingStep & 1.0 & 0.0 \\
processingDateTime & 1.0 & 0.0 \\
processingSoftware & 1.0 & 0.0 \\
softwareCreator & 1.0 & 0.0 \\
softwareName & 1.0 & 0.0 \\
softwareVersion & 1.0 & 0.0 \\
Styles & 1.0 & 0.0 \\
ParagraphStyle & 2.64 & 1.68 \\
Layout & 1.0 & 0.0 \\
Page & 1.0 & 0.0 \\
PrintSpace & 1.0 & 0.0 \\
ComposedBlock & 1.09 & 0.32 \\
TextBlock & 17.71 & 6.64 \\
TextLine & 28.67 & 4.38 \\
String & 74.27 & 17.98 \\
SP & 45.06 & 13.08 \\
GraphicalElement & 7.78 & 0.93 \\
HYP & 1.07 & 0.26 \\
TopMargin & 1.0 & 0.0 \\
LeftMargin & 1.0 & 0.0 \\
RightMargin & 1.0 & 0.0 \\
BottomMargin & 1.0 & 0.0 \\
Shape & 1.67 & 0.58 \\
Polygon & 1.67 & 0.58 \\
Illustration & Illustration & 0 \\
\hline\hline
\end{tabular}
\end{footnotesize}
\begin{quote}
\caption{Version 2}
\end{quote}
\end{figure}

\end{minipage}
\end{scriptsize}
\end{table}}








