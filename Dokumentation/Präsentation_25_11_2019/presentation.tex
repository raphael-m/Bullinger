%%%%%%%%%%%%%%%%%%%%%%%%%%%%%%%%%%%%%%%%%
% Beamer Presentation
% LaTeX Template
% Version 1.0 (10/11/12)
%
% This template has been downloaded from:
% http://www.LaTeXTemplates.com
%
% License:
% CC BY-NC-SA 3.0 (http://creativecommons.org/licenses/by-nc-sa/3.0/)
%
%%%%%%%%%%%%%%%%%%%%%%%%%%%%%%%%%%%%%%%%%

%----------------------------------------------------------------------------------------
%	PACKAGES AND THEMES
%----------------------------------------------------------------------------------------

\documentclass{beamer}

\mode<presentation> {

% The Beamer class comes with a number of default slide themes
% which change the colors and layouts of slides. Below this is a list
% of all the themes, uncomment each in turn to see what they look like.

%\usetheme{default}
%\usetheme{AnnArbor}
%\usetheme{Antibes}
%\usetheme{Bergen}
%\usetheme{Berkeley}
%\usetheme{Berlin}
%\usetheme{Boadilla}
\usetheme{CambridgeUS}
%\usetheme{Copenhagen}
%\usetheme{Darmstadt}
%\usetheme{Dresden}
%\usetheme{Frankfurt}
%\usetheme{Goettingen}
%\usetheme{Hannover}
%\usetheme{Ilmenau}
%\usetheme{JuanLesPins}
%\usetheme{Luebeck}
%\usetheme{Madrid}
%\usetheme{Malmoe}
%\usetheme{Marburg}
%\usetheme{Montpellier}
%\usetheme{PaloAlto}
%\usetheme{Pittsburgh}
%\usetheme{Rochester}
%\usetheme{Singapore}
%\usetheme{Szeged}
%\usetheme{Warsaw}

% As well as themes, the Beamer class has a number of color themes
% for any slide theme. Uncomment each of these in turn to see how it
% changes the colors of your current slide theme.

%\usecolortheme{albatross}
%\usecolortheme{beaver}
%\usecolortheme{beetle}
%\usecolortheme{crane}
\usecolortheme{dolphin}
%\usecolortheme{dove}
%\usecolortheme{fly}
%\usecolortheme{lily}
%\usecolortheme{orchid}
%\usecolortheme{rose}
%\usecolortheme{seagull}
%\usecolortheme{seahorse}
%\usecolortheme{whale}
%\usecolortheme{wolverine}

\usepackage{listings}
\usepackage{color}
\usepackage[official]{eurosym}

\definecolor{mygreen}{rgb}{0,0.6,0}
\definecolor{mygray}{rgb}{0.5,0.5,0.5}
\definecolor{mymauve}{rgb}{0.5,0,0}

\lstset{ 
  backgroundcolor=\color{white},   % choose the background color; you must add \usepackage{color} or \usepackage{xcolor}; should come as last argument
  basicstyle=\tiny,        % the size of the fonts that are used for the code
  breakatwhitespace=false,         % sets if automatic breaks should only happen at whitespace
  breaklines=true,                 % sets automatic line breaking
  captionpos=b,                    % sets the caption-position to bottom
  commentstyle=\color{mygreen},    % comment style
  deletekeywords={...},            % if you want to delete keywords from the given language
  escapeinside={\%*}{*)},          % if you want to add LaTeX within your code
  extendedchars=true,              % lets you use non-ASCII characters; for 8-bits encodings only, does not work with UTF-8
  firstnumber=1,                % start line enumeration with line 1000
  frame=single,	                   % adds a frame around the code
  keepspaces=true,                 % keeps spaces in text, useful for keeping indentation of code (possibly needs columns=flexible)
  keywordstyle=\color{blue},       % keyword style
  language=Octave,                 % the language of the code
  morekeywords={*,...},            % if you want to add more keywords to the set
  numbers=left,                    % where to put the line-numbers; possible values are (none, left, right)
  numbersep=5pt,                   % how far the line-numbers are from the code
  numberstyle=\tiny\color{mygray}, % the style that is used for the line-numbers
  rulecolor=\color{black},         % if not set, the frame-color may be changed on line-breaks within not-black text (e.g. comments (green here))
  showspaces=false,                % show spaces everywhere adding particular underscores; it overrides 'showstringspaces'
  showstringspaces=false,          % underline spaces within strings only
  showtabs=false,                  % show tabs within strings adding particular underscores
  stepnumber=1,                    % the step between two line-numbers. If it's 1, each line will be numbered
  stringstyle=\color{mymauve},     % string literal style
  tabsize=2,	                   % sets default tabsize to 2 spaces
  title=\lstname                   % show the filename of files included with \lstinputlisting; also try caption instead of title
}

%\setbeamertemplate{footline} % To remove the footer line in all slides uncomment this line
%\setbeamertemplate{footline}[page number] % To replace the footer line in all slides with a simple slide count uncomment this line

%\setbeamertemplate{navigation symbols}{} % To remove the navigation symbols from the bottom of all slides uncomment this line
}

% Anführungs- und Schlusszeichen
\newcommand
\quotes[1] {\flqq #1\frqq}

\usepackage{graphicx} % images
\usepackage{booktabs} % \toprule, \midrule and \bottomrule in tables
\usepackage{tikz}

% Sprache, Kodierung
\usepackage[german]{babel}  % Silbentrennung ("Table of Contents" --> "Inhaltsverzeichnis")
\usepackage[utf8]{inputenc}  % Umlaute
\usepackage[T1]{fontenc}

% Farben
\usepackage{hyperref}
\usepackage{wasysym}
\usepackage{listings}
\usepackage{xcolor}

% title data
\title[HBBW]{
	\quotes{Crowdsourcing}-Initiative\\ zur Digitalisierung\\
	von Heinrich Bullingers Briefwechsel\\
	\vspace*{0.3cm}
	{\scriptsize 
		Institut für Computerlinguistik\\
		\vspace*{-0.15cm}
		Institut für Schweizerische Reformationsgeschichte	
	}
}
\author{M. Volk, B. Schroffenegger}
\institute[UZH]{
	Universität Zürich\\
}
\date{\today}


\begin{document}



\begin{frame}
\titlepage
\end{frame}



\begin{frame}
\frametitle{Total \textbf{10'093} Karteikarten ...}

\vspace*{-0.3cm}
\begin{center}
\includegraphics[scale=0.055]{Bilder/HBBW_Karteikarte_00007.png}
\end{center}

\vspace*{-0.8cm}
\begin{footnotesize}
\underline{Abschätzungen:}
\begin{tiny}
	\begin{itemize}
	\item Arbeitsaufwand: 3 $\pm$ 2 min/Karte $\hat{=}$ 63 $\pm$ 42 Arbeitstage
	\item Kosten: $11'102 \pm 7'402$ CHF (22 CHF/h)
	\end{itemize}
\end{tiny}
\end{footnotesize}

\end{frame}



\begin{frame}
	\frametitle{\textbf{Inhalt}}
	\tableofcontents
\end{frame}



\section{Datenextraktion}

\subsection{OCR}


\begin{frame}
\frametitle{\textbf{OCR} (= Optical Character Recognition)}

Automatische Texterkennung (Optische Zeichenerkennung):
\begin{itemize}
\item digitaler Text $\subseteq$ digitales Bilder
$\rightarrow$ Format-Transformation (Extrakt)

\item Mustererkennung:
	\begin{itemize}
	\item Formate: PDF $\rightarrow$ XML
	\item Atome: Pixel (Raster-/Vektorgraphik) $\rightarrow$ Zeichen (Textformat)
	\item Eigenschaften: Position/Farbe $\rightarrow$ Position/Symbol
	\end{itemize}
\item Software: ABBYY FineReader
	\begin{itemize}
	\item unterstützt insbesondere Latein \& Althochdeutsch
	\item 3 Problemkarteikarten
	\item proprietär (0.01 CHF/Seite)
	\end{itemize}
\item Typische Annotationen (Elemente/Attribute/Attributwerte):
	\begin{itemize}
	\item Wort (Grösse/Position)
	\item Textzeile (Grösse/Position)
	\item Textblöcke, Seiten, ... (!)
	\end{itemize}
\end{itemize}
\end{frame}


\begin{frame}

\frametitle{\textbf{OCR-Output} (FineReader ABBYY)}

\vspace*{-0.8cm}
\begin{center}
\includegraphics[scale=0.55]{Bilder/ocr_output.png}
\end{center}

\vspace*{-0.3cm}
\begin{scriptsize}
Parsing: mit Python (xml.sax/pandas)
$\rightarrow$ ...
$\rightarrow$ Datenbank (SQLAlchemy)
\end{scriptsize}

\end{frame}

\subsection{Analysen}

\subsubsection{Karteikartengrössen}

% Karte
\begin{frame}
\frametitle{\textbf{Karteikartengrösse} (Länge $\times$ Breite)}

Stichprobe: 100 Karteikarten mit/ohne Ausreisser (4\%)

\vspace*{-0.5cm}
\begin{center}
\begin{minipage}[t]{0.45\textwidth}
\begin{center}
\includegraphics[scale=0.35]{Bilder/page_size.png}
\end{center}
\end{minipage}
\begin{minipage}[t]{0.45\textwidth}
\begin{center}
\includegraphics[scale=0.35]{Bilder/page_size_adjusted.png}
\end{center}
\end{minipage}
\end{center}
\vspace*{0cm}
\begin{center}
\begin{minipage}[t]{0.45\textwidth}
\begin{center}
\begin{table}[H]
\centering
\begin{tiny}
\begin{tabular}{lcccc}
\toprule
Seite &  $\mu_a$ &  $\sigma_a$ &  min &  max \\
\midrule
Breite	&    9661  &		975 &     4922 &     9902 \\
Höhe 	&    6837  &		690 &     3488 &     7012 \\
\bottomrule
\end{tabular}
\end{tiny}
\end{table}
\end{center}
\end{minipage}
\begin{minipage}[t]{0.45\textwidth}
\begin{center}
\begin{table}[H]
\centering
\begin{tiny}
\begin{tabular}{lcccc}
\toprule
Seite &  $\mu_b$ &  $\sigma_b$ &  min &  max \\
\midrule
Breite	&    \textbf{9860} &		11 & 9843 & 9902 \\
Höhe	&    \textbf{6978} &		10 & 6949 & 7012 \\
\bottomrule
\end{tabular}
\end{tiny}
\end{table}
\end{center}
\end{minipage}
\end{center}

\end{frame}







% Karteikartengrösse 10093
\begin{frame}

\frametitle{\textbf{Karteikartengrösse} (Länge $\times$ Breite)}

Allgemein: total 10'093 Dateien mit 259 Ausreissern ($\approx$2.6\%, orange)

\begin{center}
\vspace{-0.3cm}
\includegraphics[scale=0.52]{Bilder/pagesize_all.png}


\vspace*{-0.8cm}
\begin{equation*}
\text{Mittelwert (rot):}\quad (x_m, y_m) = (9736, 6894)
\end{equation*}
\end{center}

\end{frame}


\begin{frame}
\frametitle{\textbf{Karteikartengrösse} (Ausreisser)}

File-IDs:

\begin{tiny}
\linespread{0.5}
1, 16, 32, 33, 38, 50, 58, 63, 76, 80, 87, 92, 111, 116, 118, 128, 130, 138, 158, 161, 162, 170,
173, 186, 228, 237, 238, 246, 254, 255, 263, 266, 313, 315, 331, 332, 336, \textbf{354}, \textbf{355}, 356, 366, 368, 378, 401, 402, \textbf{426}, \textbf{432}, 439, 444, 448, 458, 469, 484, 490, 495, 502, 509, 512, 535, 541, 543, 553, 565, 569, 591, 593, 609, 618, \textbf{632}, 638, 657, 658, 665, 666, 671, 679, 680, 694, 696, 698, 707, 713, 728, 746, 762, 763, 827, 828, 832, 833, 858, 860, 861, 867, 874, 889, 916, 918, 945, 965, 985, 1004, 1009, 1034, 1042, 1045, 1048, 1054, 1079, 1089, 1091, 1097, 1099, 1101, 1102, 1125, 1128, 1129, 1143, 1150, 1169, 1176, 1209, 1213, 1238, 1240, 1245, 1267, 1268, \textbf{1273}, 1274, 1275, 1276, 1278, 1281, 1282, 1283, 1284, 1289, 1292, 1309, 1329, 1333, 1334, 1339, 1346, 1354, 1365, 1372, 1373, 1374, 1375, 1388, 1398, 1404, 1433, 1439, 1446, 1469, 1488, \textbf{1489}, 1490, 1493, 1494, 1496, 1510, 1541, 1569, 1595, 1609, 1612, 1618, 1620, 1624, 1628, 1636, 1640, 1642, 1655, 1662, 1680, 1681, 1682, 1697, 1715, 1720, 1728, 1752, 1759, 1764, 1766, 1772, 1777, 1786, 1787, 1789, 1798, 1809, 1827, 1832, 1840, 1853, 1854, 1855, 1859, 1860, 1864, 1865, 1903, 1928, 1936, 1940, 1941, 1948, 1955, 1974, 1975, 1980, 1988, 1991, 2010, 2026, 2038, 2070, \textbf{2071}, 2072, 2096, 2109, 2126, 2141, 2148, 2149, 2176, 2495, 3152, 4247, \textbf{4249}, 4250, 4685, 4721, 5198, 5570, 5649, 5828, 6071, 6234, 6426, \textbf{6680}, 6681, 6708, 6865, 7054, 7432, 7806, 7961, 7966, 8200, 8479, 8537, 8818, 9227, 9297, 9298, 9708, 9709, 9710, 9810, 9942, 10002
\end{tiny}
\end{frame}



\begin{frame}
\frametitle{\textbf{Karteikartengrösse} (Skalierung)}

\vspace*{-0.1cm}
Normierung aller Karteikarten $i\in\{1,\ldots, 10'093\}$ auf Einheitsgrösse:

\begin{itemize}
\item Seitenbreite $B_i$ $\propto$ $S_{i\alpha x}$(WIDTH$_{i\alpha x}$, HPOS$_{i\alpha x})$
\item Seitenhöhe $H_i$ $\propto$ $S_{i\alpha y}$(HEIGHT$_{i\alpha y}$, VPOS$_{i\alpha y}$)
\end{itemize}

\vspace*{-0.5cm}
\begin{center}
\begin{tikzpicture}[fill=blue!20, scale=0.8]
% Koordinaten/Beschriftungen
\draw[line width=2pt] (0,-5) -- (0,0) -- (10,0);
% \draw (-2,0) node[anchor=south]{\scriptsize \code{page}};

% Element
\fill (2,-2) -- (6,-2) -- (6,-3) -- (2,-3) -- (2,-2);

\draw[>=latex,<->] (2,0) -- (2,-2);
\draw (2,-1) node[anchor=west]{{\tiny $VPOS_\alpha$}};
\draw[>=latex,<->] (0,-2) -- (2,-2);
\draw (1,-2) node[anchor=south]{{\tiny $HPOS_\alpha$}};
\draw[>=latex,<->] (2,-3.3) -- (6,-3.3);
\draw (4,-3.3) node[anchor=north]{{\tiny $WIDTH_\alpha$}};
\draw[>=latex,<->] (6.3,-2) -- (6.3,-3);
\draw (6.3,-2.5) node[anchor=west]{{\tiny $HEIGH_\alpha$}};

\draw[red, densely dotted] (2,-3)--(2,-3.5);
\draw[red, densely dotted] (6,-3)--(6,-3.5);
\draw[red, densely dotted] (6,-3)--(6.5,-3);
\draw[red, densely dotted] (6,-2)--(6.5,-2);

\draw[dashed] (4,0)--(4,-2.5);
\draw[dashed] (0,-2.5)--(4,-2.5);

\draw (4,0) circle (2pt);
\draw (4,-2.5) circle (2pt);
\draw (0,-2.5) circle (2pt);

\draw (4,0) node[anchor=south]{$S_{\alpha x}$};
\draw (4,-2.5) node[anchor=west]{$S_\alpha $};
\draw (0,-2.5) node[anchor=east]{$S_{\alpha y}$};

\end{tikzpicture}
\end{center}

\vspace*{-0.9cm}
\begin{footnotesize}
\begin{equation*}
\text{Skalierungsfaktoren:}\quad
x_s (B_i) = \frac{\hat{B}}{B_i},\quad
y_s (H_i) = \frac{\hat{H}}{H_i}\quad
(\hat{B}=9860,\,
\hat{H}=6978).
\end{equation*}
\end{footnotesize}


\end{frame}

\subsubsection{Dateigrössen}

\begin{frame}

\frametitle{\textbf{Dateigrössen} [MB]}

Verdächtig: 10 (?) Briefe (0.99\permil)

\begin{center}
\includegraphics[scale=0.5]{Bilder/suspect_size.png}
\end{center}

\end{frame}




\subsubsection{Partitionierung}

\begin{frame}[fragile]

\frametitle{\textbf{Attribute/Werte} (Position)}

\begin{figure}[H]
	\begin{minipage}[t]{0.45\textwidth}
		\begin{center}
			\includegraphics[scale=0.19]{Bilder/Karteikarte_Original.png}
		\end{center}
	\end{minipage}
	\begin{minipage}[t]{0.45\textwidth}
		\begin{center}
			\includegraphics[scale=0.19]{Bilder/Karteikartenmasse.png}
		\end{center}
	\end{minipage}
    \caption{Messungsergebnisse (in [\%] und [px])}
    \label{fig:sample_figure}
\end{figure}

\begin{scriptsize}
\vspace{-0.3cm}
\begin{lstlisting}
% Berechnung
f = lambda l: [sum(l[:i+1]) for i, _ in enumerate(l)]  # Partialsummen
f([b*9860 for b in [0.31, 0.35, 0.34]])  # [3057, 6508, 9860]
f([h*6978 for h in [0.22, 0.29, 0.1, 0.39]])  # [1535, 3559, 4257, 6978]
f([h*6978 for h in [0.22, 0.14, 0.15, 0.25, 0.24]]) #[1535,2512,3559,5303,6978]
[1535+i*0.29*6978/4 for i in range(1,5)]  # [2041, 2547, 3053, 3559]
\end{lstlisting}
\end{scriptsize}

\end{frame}


\begin{frame}
\frametitle{\textbf{Attribute/Werte} (Position)}

Typische Werte (Mittelwerte/Standardabweichungen):


\begin{center}
\begin{tiny}
\begin{tabular}{lrrrr}
\toprule
       Value &       x &       y &    $\sigma_x$ & $\sigma_y$\\
\midrule
       Datum &   364.0 &   320.0 &  20.0 &  13.0 \\
    Absender &  3631.0 &   329.0 &  17.0 &  12.0 \\
   Empfänger &  7134.0 &   349.0 &  18.0 &  12.0 \\
   Autograph &   515.0 &  1901.0 &  19.0 &  13.0 \\
       Kopie &  3465.0 &  1903.0 &  17.0 &  12.0 \\
  Photokopie &  7120.0 &  1904.0 &  26.0 &  12.0 \\
    Standort &   437.0 &  2281.0 &  20.0 &  12.0 \\
    Standort &  3577.0 &  2283.0 &  19.0 &  12.0 \\
       Bull. &  6861.0 &  2282.0 &  31.0 &  13.0 \\
       Bull. &  6872.0 &  3284.0 &  37.0 &  18.0 \\
       Corr. &  7261.0 &  2284.0 &  31.0 &  13.0 \\
       Corr. &  7273.0 &  3285.0 &  36.0 &  17.0 \\
       Sign. &   298.0 &  2903.0 &  18.0 &  12.0 \\
       Sign. &  3442.0 &  2903.0 &  16.0 &  12.0 \\
   Abschrift &  7049.0 &  2890.0 &  25.0 &  11.0 \\
      Umfang &   397.0 &  3297.0 &  21.0 &  13.0 \\
      Umfang &  3543.0 &  3299.0 &  17.0 &  12.0 \\
     Sprache &   417.0 &  3917.0 &  22.0 &  13.0 \\
   Literatur &  3560.0 &  3904.0 &  20.0 &  13.0 \\
    Gedruckt &   448.0 &  4594.0 &  21.0 &  13.0 \\
 Bemerkungen &  3752.0 &  5691.0 &  19.0 &  15.0 \\
\bottomrule
\end{tabular}
\end{tiny}
\end{center}


\end{frame}



\begin{frame}
\frametitle{\textbf{Attribute/Werte} (Position)}

Plots (100 KK): Attributnamen, Attributwerte \& Trennlinien zw. Feldern

\begin{itemize}
\item doppelte Attributnamen
	\begin{itemize}
	\item Standort, Sign., Umfang (Autograph/Kopie)
	\item Bull. Corr. (Photokopie/Abschrift)
	\end{itemize}
\item korrupte Daten: durchschnittliche Position eines Attributnamens...
	\begin{itemize}
	\item links: undifferenziert zwischen Namen und Position
	\item rechts (unskaliert): separate Darstellung der Ausreisser
	\end{itemize}
\end{itemize}

\vspace*{-0.7cm}

\begin{center}
\begin{minipage}[t]{0.45\textwidth}
\begin{center}
\includegraphics[scale=0.35]{Bilder/ocr_attributes_1.png}
\end{center}
\end{minipage}
\begin{minipage}[t]{0.45\textwidth}
\begin{center}
\includegraphics[scale=0.35]{Bilder/ocr_attributes_2.png}
\end{center}
\end{minipage}

\end{center}

\end{frame}




\begin{frame}
\frametitle{\textbf{Funktionen}}

Beziehungen zw. Attributen/Werten und Position derselben:

\begin{itemize}
\item $(x,y) \rightarrow$ Zugehörigkeit
\item $($Attributname$, x, y) \rightarrow$ Wert oder Bezeichnung? 
\end{itemize}

\begin{center}
\begin{minipage}[t]{0.3\textwidth}
\begin{center}
\includegraphics[scale=0.25]{Bilder/get_attribute_name.png}
\end{center}
\end{minipage}
\begin{minipage}[t]{0.65\textwidth}
\begin{center}
\includegraphics[scale=0.25]{Bilder/is_attribute.png}
\end{center}
\end{minipage}

\end{center}

\end{frame}





\subsection{Aufbereitung der Daten}

\begin{frame}

\frametitle{\textbf{Aufbereitung der Daten}}

Attributnamen entfernen\\
{\footnotesize $\rightarrow$ Position und \quotes{physische} Erscheinung bekannt}

\medskip
Beseitigung von OCR-Fehlern...

\begin{footnotesize}
\begin{itemize}

\item Datum: Karteikartenordnung ausnutzen (Jahre/Monate)
\item Personen: Bullinger immer präsent (immer?); Ähnlichkeitsvergleiche (N-Gramme)
\item \textit{((Ortschaften: Ähnlichkeitsvergleiche gegen bekannte (Schweizer-) Städtenamen))}
\item Sprache: Deutsch, Lateinisch, Griechisch. Spracherkennungstool \textit{\quotes{langid}}
\item Referenzen (Photokopie/Abschrift): Einschränkungen auf nummerische Zeichen
\item Standort: Präsenz bestimmter \quotes{keywords} (Schreibmaschine?)
\item Bemerkung: insbesondere Bindestriche entfernen
\item Allgemein: exotische Zeichen entfernen; nach Zahlen/Buchstaben filtern, typische Fehler korrigieren. Beispiele:

{\footnotesize \{0, O, o, $^\circ$\}, \{B, 8, $\beta$, g\}, \{I, l, i, 1, $\vert$\}, \{ii, ü\}, \{S, \$\}, \\
\{7, ?, y\}, \{A, 4\}, \{£, E, \euro{}\}, ...}



%s
\end{itemize}
\end{footnotesize}

\end{frame}




\subsubsection{Heuristiken}


\begin{frame}

\frametitle{\textbf{Heuristiken}}

\begin{itemize}

\item Schreibmaschinen-Schlüsselwörter
\begin{scriptsize}
\begin{itemize}
\item[1.] StA, StB, Ms., Nr., Hr., (ZB), ... (?)
\item[2.] Zürich, Chur, Basel, St.Gallen, Genf, London, Den Haag, ...
\end{itemize}
\end{scriptsize}

\item Autograph/Kopie
\begin{itemize}
\item Beispiele (Standort/Signatur)
	\begin{itemize}
	\item Zürich StA, E II 360,85
	\item Zürich ZB (ZZB), Ms S 67, 84
	\end{itemize}
\item Regeln:
	\begin{itemize}
	\item \{E, £, \euro{}\} $\in$ Signatur[0:4]
	$\quad\Rightarrow\quad$ Standort: \quotes{Zürich StA}
	\item \{A, 4\} $\in$ Standort[-4:-1]
	$\quad\Rightarrow\quad$ Standort: \quotes{Zürich StA}
	\item \{22, 2Z, Z2, 22\} $\in$ Standort[0:4]
	$\quad\Rightarrow\quad$ \quotes{Zürich ZB} (ZZB)
	\item \{Z, 2, 7, $\hat{}$\,\} $\in$ Standort[0:4]
	$\quad\Rightarrow\quad$ \quotes{Zürich}
\end{itemize}

\item typische Fehler:
	\begin{itemize}
	\item \{U, T, X\} statt II
	\item Kombinationen [Zürich StA EU 365,757]
	\end{itemize}
\end{itemize}

\item Junk?
\begin{itemize}
\item Buchstaben/Zeichen-Verhältnis; Durchschnittliche Wortlänge; Einzelzeichen; ...
\end{itemize}




\end{itemize}

\end{frame}









\subsubsection{Silbentrennung}

\begin{frame}[fragile]

\frametitle{\textbf{Silbentrennungen}}

Viele Bindestriche: Bspw. \quotes{Baselines} [['Ein', 'Bei-'], ['spiel']]

\begin{lstlisting}
if (lines[i][-1][-1] == '-'):
	 lines[i][-1][-1] = ''
	 concat(lines[i][-1], lines[i+1][0])
\end{lstlisting}
\vspace*{-0.5cm}
$\quad\rightarrow$ Erfolgsquote bei ca. 40\% (9:13 über 100)

\begin{center}
\includegraphics[scale=0.55]{Bilder/Bindestriche.png}
\end{center}

Lösung (Parser):

\begin{footnotesize}
\begin{lstlisting}
d, v = dict(), None
for a in attr.getNames(): d[a] = attr.getValue(a)
if ("SUBS_TYPE" in d) and ("SUBS_CONTENT" in d):
	 if (d["SUBS_TYPE"] == "HypPart1"):
		  v = d["SUBS_CONTENT"]
else: v = d["CONTENT"]
\end{lstlisting}

\end{footnotesize}

\end{frame}






\section{Datenbank}


\subsubsection{Anforderungen/Designprinzipien}

\begin{frame}

\frametitle{\textbf{Datenbank}}

\begin{itemize}

\item
Herausforderungen:
	\begin{itemize}
	\item Vandalismus
	\item Redundanz\\
	\end{itemize}

\item
Lösungen:
	\begin{itemize}
	\item Benutzerkonten
	\item Normalisierung\\
	\end{itemize}
	
\item
Normalformen

\begin{itemize}
\item[1.] keine mehrwertigen Attribute (Sortier-/Selektierbarkeit)
\item[2.] keine partielle Abhängigkeiten von Nicht-Schlüsselattributen vom Gesamtschlüssel (weniger Redundanzen/Updateanomalien)
\item[3.] kein Nichschlüsselattribut hängt transitiv von einem Schlüsselkandidaten ab (logische, monothematische DB-Struktur)
\end{itemize}

\end{itemize}

\end{frame}




\subsubsection{Relationen}

\begin{frame}
\frametitle{\textbf{Relationen}}
Primärschlüssel: Karten Nummer (\textbf{ID})

\begin{footnotesize}
\begin{itemize}
\item Benutzer(Name, E-Mail, Passwort-Hash)
\item Kartei(\textbf{ID}, Rezensionen, Status, Pfad\_OCR, Pfad\_IMG, Pfad\_PDF)
\item Datum(\textbf{ID}, J1, M1, D1, J2, M2, D2, Bemerkung, Benutzer, Zeit)
\item Person(\underline{IDP}, NN, VN, Ort, Titel, Benutzer, Zeit)
\item Absender(\textbf{ID}, \underline{IDP}, Bemerkung, Benutzer, Zeit) 
\item Empfänger(\textbf{ID}, \underline{IDP}, Bemerkung, Benutzer, Zeit)
\item Autograph(\textbf{ID}, Standort, Signatur, Umfang, Benutzer, Zeit)
\item Kopie(\textbf{ID}, Standort, Signatur, Umfang, Benutzer, Zeit)
\item Photokopie(\textbf{ID}, Standort, Bull. Corr, Blatt, Seite, Benutzer, Zeit)
\item Abschrift(\textbf{ID}, Standort, Bull. Corr, Blatt, Seite, Benutzer, Zeit)
\item Sprache(\textbf{ID}, Sprache, Benutzer, Zeit)
\item Literatur(\textbf{ID}, Literatur, Benutzer, Zeit)
\item Gedruckt(\textbf{ID}, Gedruckt, Benutzer, Zeit)
\item Bemerkung(\textbf{ID}, Bemerkung, Benutzer, Zeit)
\end{itemize}
\end{footnotesize}

\end{frame}




\section{Interface}
\subsubsection{Live Demo}



\section{Tools}

\begin{frame}
\frametitle{Werkzeuge}

\begin{itemize}
\item Frontend:
	\begin{itemize}
	\item HTML, CSS
	\item JS/JQuery
	\end{itemize}
\item Backend (Python):
	\begin{itemize}
	\item Server (Flask/Jinja2/Werkzeug)
	\item Datenbank (SQLAlchemy)
	\item Parser (xml.sax)
	\item Sprachidentifikation (LangID)
	\item Tabellenmanagement (pandas)
	\item Dictionaries (defaultdict)
	\item Dateimanagement (os)
	\item pdf2image
	\end{itemize}

\end{itemize}

\end{frame}


\begin{frame}
\frametitle{Interface}

\includegraphics[scale=0.23]{Bilder/Interface.png}
\end{frame}


%\begin{frame}
%\frametitle{Übersicht}
%\tableofcontents
%\end{frame}
%
%\section{N-Gramm-Analysen}
%
%\begin{frame}
%\frametitle{N-Gramm-Analyse}
%\begin{itemize}
%\item Textfragmentierung (Bsp. ``Hello World'')
%	\begin{itemize}
%		\item Monogramme (1-Gramme): $\{ H, e, l, l, o, W, o, r, l, d  \}$
%		\item Bigramme (2-Gramme): $\{ \_H, He, el , ll, lo, o\_, \_W, Wo, or, rl, ld \}$
%		\item Trigramme (3-Gramme): $\{ \_\_H, \_He, Hel, ell, llo, ... \}$
%		\item ...\\
%		$\rightarrow$ auf Basis von Buchstaben, Wörter, (Sätze), ...
%	\end{itemize}
%\item Wahrscheinlichkeiten einzelner Wort- oder Buchstabenfolgen
%\item Distanzfunktion: Dice-Koeffizient
%	\begin{itemize}
%		\item $a$, $b$ Dokumente
%		\item $T(x)$: n-Gramme des Dokuments $x$
%	\end{itemize}
%	
%	\begin{equation*}
%		d(a,b) = \frac{2\vert T(a) \cap T(b) \vert }{\vert T(a)\vert + \vert T(b)\vert} \in [0,1]
%	\end{equation*}
%\end{itemize}
%
%\end{frame}
%
%
%\begin{frame}
%\frametitle{Beispiele}
%\begin{itemize}
%	\item Dokumente
%		\begin{itemize}
%			\item $a$: ``hello world''
%			\item $b$: ``hellooo''
%		\end{itemize}
%	\item $n=1$
%		\begin{itemize}
%			\item $T(a) = \{(h,1), (e,1), (l, 3), (o,2), (w,1), (r,1), (d,1)\}$
%			\item $T(b) = \{(h,1), (e,1), (l, 2), (o,3)\}$
%			\item $d_z(a,b) = 2(1+1+2+2) / (10+7) = 12/17 = 71\%$
%			\item $d_w(a,b) = 2\times (0) / (3) = 0\%$
%		\end{itemize}
%	\item Varianten
%		\begin{itemize}
%			\item Gross- \& Kleinschreibung (nicht) berücksichtigen
%			\item mit/ohne Satzzeichen
%			\item mit/ohne/nur Zahlen
%			\item nur die 300 häufigsten n-Gramme berücksichtigen (Speicher!)
%			\item Mittelwerte für verschiedene n bilden
%		\end{itemize}
%\end{itemize}
%\end{frame}


\end{document} 