\documentclass[12pt]{article}

\usepackage{geometry}
\geometry {
	a4paper,
	left=1cm,
	right=1cm,
	top=1cm,
	bottom=1cm
}

% Sprache, Kodierung
\usepackage[german]{babel}  % Silbentrennung ("Table of Contents" --> "Inhaltsverzeichnis")
\usepackage[utf8]{inputenc}  % Umlaute
\usepackage[T1]{fontenc}

% Farben
\usepackage{xcolor}
\usepackage[colorlinks=true, urlcolor=mailblue]{hyperref}

\definecolor{mailblue}{RGB}{6,69,173}
\definecolor{mygreen}{rgb}{0,0.6,0}
\definecolor{deepgreen}{rgb}{0,0.7,0}
\definecolor{mygray}{rgb}{0.5,0.5,0.5}
\definecolor{mymauve}{rgb}{0.58,0,0.82}
\definecolor{deepred}{rgb}{0.6,0,0}

\usepackage{array}
\usepackage{tikz}
\usetikzlibrary{positioning,automata}

% Abkürzungsverzeichnis
\usepackage{suffix}
\usepackage{xstring}
\usepackage[printonlyused, withpage]{acronym}
    
\usepackage{filecontents}
\usepackage{datatool}
\usepackage{amsmath}
\usepackage{amssymb}
\usepackage{mathabx}  % diameter (avg) symbol
\usepackage{textcomp}
\usepackage{tikz}
\usepackage{pgfgantt}
\usepackage{float}
\usepackage{calc}
\usepackage[symbol]{footmisc}
\usepackage{graphicx}

\usepackage{pgfplots}
\pgfplotsset{width=7cm, compat=1.5, table/search path={Data}}
% Preamble: \pgfplotsset{,compat=1.16}

\usepackage{caption}

% Main-Title
\newcommand\maintitle[2]{
	\begin{center}
		\textbf{{\Huge\textsc{#1}}}\\
		\vspace{-0.2cm}
		\rule{0.7\textwidth}{3pt}\\
		\vspace{0.2cm}
		{\large #2}\\
	\end{center}
}

% subsubsubsection
\newcommand
\ssss[1] {\medskip\noindent\textsc{\textbf{#1\medskip}}}

% Header
\newcommand\newtitle[1]{
	%\textrule{0.5pt}
	
	%\vspace{-0.3cm}
	{\noindent\bigskip\Large\bf\scshape #1}
	
	\vspace{-0.7cm}
	\textrule{0.5pt}
	\vspace{-0.5cm}
}

% Anführungs- und Schlusszeichen
\newcommand
\quotes[1] {\flqq #1\frqq}

% Horizontale Linie
\usepackage[normalem]{ulem}
\newcommand
\textrule[1] {
	\noindent
	\rule{\textwidth}{#1}
}

% Abbildung (Caption)
\newcommand
\abb[1] {
	\begin{quotation}
		\vspace{-0.8cm}
		\noindent\caption{#1}
	\end{quotation}
}

% Erster Buchstabe eines neuen Kapitels
\usepackage{lettrine}
\newcommand\fl[1] {
	\lettrine[lraise=0.05, nindent=0em, slope=-.5em]{#1}{}{}
}
\newcommand\fls[1] {
	\lettrine[lines=2]{#1}{}{}
}

% Listen
\renewcommand{\labelitemi}{$\blacktriangleright$}
\renewcommand{\labelitemii}{$\diamond$}
\renewcommand{\labelitemiii}{$\triangleright$}
\renewcommand{\labelitemiv}{$\ast$}
\newcommand{\litem}{\text{-}}  % Literaturverzeichnis
\usepackage{enumitem}

% TOC
\usepackage{tocloft}
\setlength\cftparskip{0pt}
\setlength\cftbeforesecskip{3pt}
\setlength\cftaftertoctitleskip{6pt}

% Head/Foot
%\usepackage{fancyhdr}
%\usepackage{extramarks}
%\pagestyle{fancy}
%\fancyhf{} % alle Kopf- und Fusszeilenfelder bereinigen
%\fancyhead[L]{\textsc{\firstleftmark}}
%\fancyhead[C]{}
%\fancyhead[R]{\slshape\rightmark}
%\renewcommand\sectionmark[1]{
%	\markboth{\thesection\ #1}{}
%}
%\fancyfoot[C]{\thepage} % Seitenzahl
%%\renewcommand{\footrulewidth}{0.4pt}
%\renewcommand\headrule{{
%	\color{black}
%	\hrule height 2pt width\headwidth
%	\vspace{1pt}
%	\hrule height 1pt width\headwidth
%	\vspace{-4pt}
%}}

\DeclareCaptionFont{CaptionFontSize}{\fontsize{10pt}{12pt}\selectfont}
\captionsetup{font=CaptionFontSize}
		

\newcommand{\code}[1]{\texttt{#1}}

% Code
\usepackage{listings}  % Darstellung von Code mit Syntaxhighlighting
\usepackage{setspace}
\usepackage{color}

\DeclareFixedFont{\ttb}{T1}{txtt}{bx}{n}{8} % for bold
\DeclareFixedFont{\ttm}{T1}{txtt}{m}{n}{8}  % for normal

\lstset{
  backgroundcolor=\color{white},   % choose the background color; you must add \usepackage{color} or \usepackage{xcolor}; should come as last argument
  basicstyle=\ttm,        % the size of the fonts that are used for the code
  breakatwhitespace=false,         % sets if automatic breaks should only happen at whitespace
  breaklines=true,                 % sets automatic line breaking
  captionpos=b,                    % sets the caption-position to bottom
  commentstyle=\color{mygreen},    % comment style
  deletekeywords={...},            % if you want to delete keywords from the given language
  escapeinside={\%*}{*)},          % if you want to add LaTeX within your code
  extendedchars=true,              % lets you use non-ASCII characters; for 8-bits encodings only, does not work with UTF-8
  firstnumber=1,                % start line enumeration with line 1000
  frame=tb,	                   % adds a frame around the code
  keepspaces=true,                 % keeps spaces in text, useful for keeping indentation of code (possibly needs columns=flexible)
  keywordstyle=\color{blue},       % keyword style
  language=Python,                 % the language of the code
  morekeywords={*,...},            % if you want to add more keywords to the set
  otherkeywords={self, None, True, False, with},  
  numbers=left,                    % where to put the line-numbers; possible values are (none, left, right)
  numbersep=5pt,                   % how far the line-numbers are from the code
  numberstyle=\tiny\color{mygray}, % the style that is used for the line-numbers
  rulecolor=\color{black},         % if not set, the frame-color may be changed on line-breaks within not-black text (e.g. comments (green here))
  showspaces=false,                % show spaces everywhere adding particular underscores; it overrides 'showstringspaces'
  showstringspaces=false,          % underline spaces within strings only
  showtabs=false,                  % show tabs within strings adding particular underscores
  stepnumber=1,                    % the step between two line-numbers. If it's 1, each line will be numbered
  stringstyle=\color{mymauve},     % string literal style
  tabsize=2,	                   % sets default tabsize to 2 spaces
  title=\lstname,                   % show the filename of files included with \lstinputlisting; also try caption instead of title
  emph={__init__, @staticmethod},          % Custom highlighting
  emphstyle=\ttb\tiny\color{deepred},    % Custom highlighting style
  stringstyle=\color{mygreen},
}


% Syntaxhighlighting for XML-code
% Source: https://tex.stackexchange.com/questions/10255/xml-syntax-highlighting
\definecolor{gray}{rgb}{0.4,0.4,0.4}
\definecolor{darkblue}{rgb}{0.0,0.0,0.6}
\definecolor{cyan}{rgb}{0.0,0.6,0.6}

\lstdefinelanguage{myXML}{
  morestring=[b]",
  morestring=[s]{>}{<},
  morecomment=[s]{<?}{?>},
  stringstyle=\color{black},
  identifierstyle=\color{darkblue},
  keywordstyle=\color{cyan},
  morekeywords={xmlns,version,type}
}

\lstset{
  literate={ö}{{\"o}}1
           {ä}{{\"a}}1
           {ü}{{\"u}}1
}

% Links
\PassOptionsToPackage{hyphens}{url}
\usepackage{hyperref}
\hypersetup{
    pdftoolbar=true,
    pdfmenubar=true,
    pdffitwindow=false,
    pdfstartview={FitH},
    pdftitle={title},
    pdfauthor={Bernard Schroffenegger},
    pdfsubject={Korpusversionen},
    pdfcreator={Bernard Schroffenegger},
    pdfproducer={Bernard Schroffenegger},
    pdfkeywords={Korpus} {Korpora} {Vergleich} {Versionen} {Werkzeug},
    pdfnewwindow=true,
    colorlinks,
    linkcolor={blue!50!black},
    citecolor={blue!50!black},
    urlcolor={blue!80!black},
	filecolor=magenta
}


% \usepackage{breakurl}

% Fussnoten
\renewcommand{\thefootnote}{\arabic{footnote}}

% Symbol über und unter "=":
% https://tex.stackexchange.com/questions/123219/writing-above-and-below-a-symbol-simultaneously, 6.5.2019
\usepackage{stackengine}
\newcommand\stackequal[2]{%
  \mathrel{\stackunder[2pt]{\stackon[4pt]{$\Rightarrow$}{$\scriptscriptstyle#1$}}{%
  $\scriptscriptstyle#2$}}}

% Tabellen
\usepackage{multirow}

\tikzstyle{bag} = [align=center]

% Bilder
\usepackage{caption}
\usepackage{subcaption}

% Tabellen
\usepackage{booktabs}  % toprule, midrule, bottomrule, \cmidrule{...}, ...